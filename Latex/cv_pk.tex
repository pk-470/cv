\documentclass[11pt,a4paper,sans]{moderncv} % Font sizes: 10, 11, or 12; paper sizes: a4paper, letterpaper, a5paper, legalpaper, executivepaper or landscape; font families: sans or roman

\moderncvstyle{classic} % CV theme - options include: 'casual' (default), 'classic', 'oldstyle' and 'banking'
\moderncvcolor{black} % CV color - options include: 'blue' (default), 'orange', 'green', 'red', 'purple', 'grey' and 'black'

\usepackage[left=0.2in,right=0.8in,top=0.6in,bottom=0.6in]{geometry}  % Reduce document margins
\setlength{\hintscolumnwidth}{0.8in} % Uncomment to change the width of the dates column
%\setlength{\makecvtitlenamewidth}{10cm} % For the 'classic' style, uncomment to adjust the width of the space allocated to your name


% Remove black lines on the side and increase the vertical spacing between the sections
\makeatletter
\RenewDocumentCommand{\section}{sm}{%
	\par\addvspace{0.25in}% vertical spacing
	\phantomsection{}
	\addcontentsline{toc}{section}{#2}%
	\parbox[t]{\hintscolumnwidth}{\strut\raggedleft\raisebox{\baseletterheight}{\rule{\hintscolumnwidth}{0.6ex}}}% black lines
	\hspace{\separatorcolumnwidth}%
	\parbox[t]{\maincolumnwidth}{\strut\sectionstyle{\textbf{#2}}}%
	\par\nobreak\addvspace{2ex}\@afterheading
}
\makeatother


% CV Title
\renewcommand*{\namefont}{\fontsize{26}{32}\mdseries\upshape}
\firstname{\hspace{0.4in} Parmenion}
\familyname{Koutsogeorgos} 

\address{Brouwersweg 100, 6216 EG}{Maastricht, NL}
\mobile{(+30) 6909156296}

\renewcommand{\emaillink}[1]{#1}
\email{\href{mailto:parmenkouts99@gmail.com}{parmenkouts99@gmail.com}}

\social[linkedin][linkedin.com/in/parmenion-koutsogeorgos-pk470/]{Parmenion Koutsogeorgos}

\social[github][github.com/pk-470]{pk-470}


% Change vertical alignment of name
\usepackage{etoolbox}
\patchcmd{\makecvhead}{{minipage}[b]}{{minipage}[c]}{}{}
\patchcmd{\makecvhead}{{tabular}[b]}{{tabular}[c]}{}{}


\nopagenumbers


\begin{document}


\makecvtitle


\cvitem[0.8em]{}{
	I am a Cambridge Mathematics graduate, currently on my second year of a two year master's in Artificial Intelligence at Maastricht University. After finishing my bachelor's and first master's in Mathematics, I started working as a math tutor. At the same time, I independently developed my programming skills in Python and took online courses on the basics of Artificial Intelligence. Having now gained strong theoretical foundations on A.I. through my current master's, I am currenctly looking for an A.I. related internship in order to get some practical experience in the field.
}


\section{Education}

\cventry[0.8em]{2022 -}{Master's in Artificial Intelligence}{Maastricht University}{}{}{
	So far I have taken courses in topics such as Search Algorithms, Deep Learning, Autonomous Robotics Systems, Computer Vision, Data Mining, Text Mining and Natural Language Processing. Additionally, as a part of my degree, I participated in two research group projects.
}

\cventry[0.8em]{2017 - 2021}{Master of Mathematics \normalfont{(bachelor's \& integrated master's)}}{University of Cambridge}{}{67/100}{
	In my two final years of study I mainly focused on Logic and Combinatorics, taking courses such as: Model Theory (on which I wrote my master's thesis), Topics in Combinatorics, Ramsey Theory, Additive Combinatorics, Logic and Set Theory, Graph Theory. Other topics I have taken include: Differential Equations, Statistics, Linear Algebra, Number Theory, Group/Ring/Module Theory, Real and Complex Analysis, Topology, Probability and Measure Theory, Linear Analysis, Differential Geometry, Riemman Surfaces.
}

\cventry[0.8em]{2014 - 2017}{Apolytirion of Geniko Lykeio \normalfont{(high-school diploma)}}{Mandoulides Schools}{}{19.9/20}{
	Due to my strong performance in school exams as well as extracuriccular maths and physics competitions (most notably winning a bronze medal in the 18th Junior Balkan Mathematical Olympiad, a gold medal in the 31st Greek Junior National Mathematical Olympiad and the first place in the 25th Panhellenic Physics Competition) I received a full scholarship throughout my years of study in Mandoulides Schools.
}


\section{Previous work experience}

\cvitem[0.8em]{2023 -}{\textbf{Teaching Assistant}:
	I am currently working as a Teaching Assistant at Maastricht University, teaching various mathematics courses.
}

\cvitem[0.8em]{2021 -}{\textbf{Private Tutor}:
	I am offering private lessons in mathematics for university, high-school and math competitions.
}


\section{Computer skills}

\cvitem[0.8em]{\textbullet}{
	\textbf{Python} (competent): I have used Python extensively for my master's in Artificial Intelligence as well as multiple personal projects. In particular, I have experience working with libraries such as NumPy, Pandas, Matplotlib, TensorFlow and PyTorch.
}

\cvitem[0.8em]{\textbullet}{
	\textbf{MATLAB} (competent): I have used MATLAB extensively in various projects for my degree in Mathematics.
}

\cvitem[0.8em]{\textbullet}{
	\textbf{Linux} (intermediate): I am familiar with Ubuntu and I have written a number of Bash scripts to automate various tasks or customize Ubuntu's behaviour and appearance.
}

\cvitem[0.8em]{\textbullet}{
	\textbf{HTML} and \textbf{CSS} (intermediate): I have learned some HTML and CSS through the online web development course ``The Odin Project".
}

\cvitem[0.8em]{\textbullet}{
	\textbf{\LaTeX} \ (competent): I regularly use Latex to write technical and non-technical documents.
}


\section{Selected projects from my A.I.\ master's:}

\cventry[0.8em]{2023}{Action and Path Planning for Environment-Aware and Collision-Free Object Manipulation}{}{}{}{
	The aim of this group project was to automate the manipulation of small objects by a robotic arm. This task involves three main components:
	\begin{itemize}
		\item[\textbullet] building a representation of the environment using 3D computer vision techniques;
		\item[\textbullet] identifying optimal grasping position for each object;
		\item[\textbullet] planning the path of the robotic arm towards the target position.
	\end{itemize}
}


\cventry[0.8em]{2022 - 2023}{Automatic Music Sheet Page Turner}{}{
	\faGithub\ \href{https://github.com/csotogd/Music-Score-Localization-2.0}{\protect\path{csotogd/Music-Score-Localization-2.0}}}{}{
	Our group developed an algorithm with the aim to localize a recorded song snippet as played by a human within the music score of the entire song. Our algorithm is based on the one developed for the app Shazam, while also including techniques such as Monte-Carlo Robot Localization.
}

\cventry[0.8em]{2022}{Impasse AI}{}{
	\faGithub\ \href{https://github.com/pk-470/impasse-ai}{\protect\path{pk-470/impasse-ai}}}{}{
	I created a search engine for the game Impasse by Mark Steere which uses Alpha-Beta search along with move ordering, iterative deepening and a transposition table. When matched with the engines of my classmates in a tournament my engine came 3rd overall, surpassing many engines which could search deeper within the allowed time due to being written in languages faster than Python.
}


\section{Selected computational projects from my Mathematics bachelor's:}

\cventry[0.8em]{2020}{Graph Colouring}{}{}{}{
	I studied and implemented on MATLAB various techniques for efficiently bounding the chromatic number of a graph.
}

\cventry[0.8em]{2019}{Simulation of Random Samples	from Parametric Distributions}{}{}{}{
	I used MATLAB to study and visualise the behaviour of random variables sampled from various distributions (exponential, gamma, normal, chi-squared).
}

\cventry[0.8em]{2018}{Ordinary Differential Equations}{}{}{}{
	I used MATLAB to study and compare three methods for solving ODEs (Leapfrog, Euler, RK4).
}


\section{Essays/Presentations}

\cventry[0.8em]{2021}{NIP Theories and O-minimality}{Part III, University of Cambridge}{
	\faGithub\ \href{https://github.com/pk-470/nip-theories}{\protect\path{pk-470/nip-theories}}}{}{
	As a part of my master's in Mathematics I worked on an expository essay on Model Theory, a branch of Mathematical Logic. This essay required extensive independent research and study of multiple papers, which I then had to condense and combine into a coherent and focused text.
}

\cventry[0.8em]{2016}{Inversion: Properties and Applications}{8th International Week Dedicated to Maths}{Thessaloniki}{}{
	While still being a student in Mandoulides Schools I independently worked on and presented a project on the properties of geometric inversion and how it can be used to produce imaginative solutions to very hard competition problems.
}


\section{Extracurricular projects and education}

\cventry[0.8em]{2022}{CS50’s Introduction to Artificial Intelligence with Python}{}{
	\faGithub\ \href{https://github.com/pk-470/cs50-ai-with-python}{\protect\path{pk-470/cs50-ai-with-python}}}{}{
	In order to familiarize myself with the basics of Artificial Intelligence and also improve my skills in Python I completed the online course ``CS50’s Introduction to Artificial Intelligence with Python'' offered by HarvardX.
}

\cventry[0.8em]{2022}{ACM AUTh Days of Coding}{}{
	\faGithub\ \href{https://github.com/pk-470/acm-auth-days-of-coding}{\protect\path{pk-470/acm-auth-days-of-coding}}}{}{
	I took part in the online team programming competition ``Days of Coding'' organised by the ACM team from the Aristotle University of Thessaloniki. Our team came first among the 30 teams participating.}

\cventry[0.8em]{2022}{Supermod}{}{
	\faGithub\ \href{https://github.com/pk-470/Supermod}{\protect\path{pk-470/Supermod}}}{}{
	In order to teach myself Python I created a Discord bot whose duties include fetching and manipulating data from Google spreadsheets, extracting and formatting data from messages, interacting with the users and making posts periodically.
}


\section{Awards/Competitions}

\cventry[0.8em]{2016}{33rd Greek National Mathematical Olympiad}{Hellenic Mathematical Society}{}{\newline Silver Medal}{}

\cventry[0.8em]{2016}{Kangaroo Hellas Mathematical Competition}{Kangourou Sans Frontieres}{}{\newline Distinction}{}

\cventry[0.8em]{2015}{American Mathematics Competition - AMC 10}{Mathematical Association of America}{}{\newline Distinction}{
	Arrived in top 2.5\% internationally.
}

\cventry[0.8em]{2015}{4th Junior European Mathematical Cup}{Young Gifted Mathematicians Marin Getaldić}{}{\newline Third Prize}{}

\cventry[0.8em]{2015}{32nd Greek National Mathematical Olympiad}{Hellenic Mathematical Society}{}{\newline Silver Medal}{}

\cventry[0.8em]{2015}{25th Panhellenic Physics Competition}{Hellenic Society for Physics, Science and Education}{}{\newline 1st Place}{}

\cventry[0.8em]{2014}{18th Junior Balkan Mathematical Olympiad}{Union of Mathematicians of Macedonia}{}{\newline Bronze Medal}{}

\cventry[0.8em]{2014}{31st Greek Junior National Mathematical Olympiad}{Hellenic Mathematical Society}{}{\newline Gold Medal}{}

\cventry[0.8em]{2013}{30th Greek Junior National Mathematical Olympiad}{Hellenic Mathematical Society}{}{\newline Silver Medal}{}


\section{Languages}

\cvitem[0.8em]{}{
	Greek (mother tongue), English (proficient), German (elementary).
}


\section{Additional information}

\cvitem[0.8em]{2018}{
	I was a member of the Emmanual College Boatclub in Cambridge and I rowed in the 2018 Fairbairn Cup.
}

\cvitem[0.8em]{2017 -}{
	For the past few years I have been playing the electric bass (mainly) and the guitar.
}


\end{document}